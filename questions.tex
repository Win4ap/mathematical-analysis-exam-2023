\documentclass[12pt]{article}
\usepackage[utf8]{inputenc}
\usepackage[english,russian]{babel}
\usepackage{graphicx}
\usepackage{amsmath}
\usepackage{amsthm}
\usepackage{caption}
\usepackage{dsfont}
\usepackage{tikz}
\usepackage{amssymb}
\usepackage{subcaption}
\usepackage{imakeidx}
\usepackage{hyperref}
\usepackage[russian]{cleveref}
\usepackage[a4paper,left=15mm,right=15mm,top=30mm,bottom=20mm]{geometry}
\parindent=0mm
\parskip=3mm
\DeclareRobustCommand{\divby}{%
\mathrel{\text{\vbox{\baselineskip.65ex\lineskiplimit0pt\hbox{.}\hbox{.}\hbox{.}}}}%
}

\makeindex
\pagestyle{empty}
\title{Матанализ}
\author{Канта контроль}
\date{\today}
\begin{document}
\maketitle
\large

1. \textbf{Множества: упорядоченная пара, декартово произведение, операции над множествами. Правила де Моргана.}

2. \textbf{Отношения: область определения, область значений, обратное отношение, композиция отношений, свойства, примеры.}

3. \textbf{Аксиомы вещественных чисел. Математическая индукция. Существование наибольшего и наименьшего элемента в конечном множестве. Следствия.}

4. \textbf{Принцип Архимеда. Следствия. Наибольший элемент в множестве целых чисел. Существование целой части числа.}

5. \textbf{! Супремум и инфимум. Определение и теорема существования. Характеристика супремума.}

6. \textbf{! Теорема о вложенных отрезках. Существенность условий.}

7. \textbf{! Монотонные и ограниченные последовательности. Два определения предела и их равносильность. Примеры.}

8. \textbf{! Простейшие свойства пределов последовательностей (единственность предела, предельный переход в неравенстве, ограниченность).}

9. \textbf{! Теорема о стабилизации знака и теорема о двух милиционерах. Следствия.}

10. \textbf{! Предел монотонной последовательности.}

11. \textbf{Арифметические свойства пределов последовательности.}

12. \textbf{! Бесконечные пределы. Бесконечно большие. Связь между бесконечно малыми и бесконечно большими. Аналоги теорем для бесконечных пределов.}

13. \textbf{Арифметические действия в $\overline{\mathbb{R}}$. Примеры.}

14. \textbf{Неравенство Бернулли.}

15. \textbf{! Определение экспоненты и числа e.}

16. \textbf{Свойства экспоненты.}

17. \textbf{Формула для экспоненты суммы (с леммой).}

18. \textbf{Сравнение скорости возрастания последовательностей $n^k$, $a^n$, $n!$ и $n^n$.}

19. \textbf{Теорема Штольца (для неопределенности $\frac{\infty}{\infty}$). Сумма m-ых степеней натуральных чисел.}

20. \textbf{Теорема Штольца (для неопределенности $\frac{0}{0}$).}

21. \textbf{Подпоследовательности (определение и простейшие свойства). Теорема о стягивающихся отрезках. }

22. \textbf{! Теорема Больцано–Вейерштрасса (в том числе и случай неограниченной последовательности).}

23. \textbf{! Фундаментальные последовательности. Свойства. Критерий Коши.}

24. \textbf{Верхний и нижний пределы. Частичные пределы. Связь между ними.}

25. \textbf{Характеристика верхних и нижних пределов с помощью $N$ и $\epsilon$. Сохранение неравенств для верхних и нижних пределов.}

26. \textbf{! Сходимость рядов. Необходимое условие сходимости рядов. Примеры.}

27. \textbf{Простейшие свойства сходящихся рядов.}

28. \textbf{Окрестности и проколотые окрестности. Предельные точки множества.}

29. \textbf{! Определения предела функций в точке. Простейшие свойства.}

30. \textbf{! Равносильность определения предела по Коши и по Гейне.}

31. \textbf{Свойства функций, имеющих предел.}

32. \textbf{Арифметические действия с пределами.}

33. \textbf{! Теорема о предельном переходе в неравенствах. Теорема о двух милиционерах.}

34. \textbf{! Критерий Коши для предела функций.}

35. \textbf{Левый и правый пределы. Предел монотонной функции.}

\end{document}
