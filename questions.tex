\documentclass[12pt]{article}
\usepackage[utf8]{inputenc}
\usepackage[english,russian]{babel}
\usepackage{graphicx}
\usepackage{amsmath}
\usepackage{amsthm}
\usepackage{caption}
\usepackage{dsfont}
\usepackage{tikz}
\usepackage{amssymb}
\usepackage{subcaption}
\usepackage{imakeidx}
\usepackage{hyperref}
\usepackage[russian]{cleveref}
\usepackage[a4paper,left=15mm,right=15mm,top=30mm,bottom=20mm]{geometry}
\parindent=0mm
\parskip=3mm
\DeclareRobustCommand{\divby}{%
\mathrel{\text{\vbox{\baselineskip.65ex\lineskiplimit0pt\hbox{.}\hbox{.}\hbox{.}}}}%
}

\makeindex
\pagestyle{empty}
\title{Матанализ}
\author{Канта контроль}
\date{\today}
\begin{document}
\maketitle
\large

1. \textbf{Множества: упорядоченная пара, декартово произведение, операции над множествами. Правила де Моргана.}\\
Множество - какой-то набор элементов. Для любого элемента можно сказать принадлежит множеству или нет.\\
$A \subset B$, то есть $\forall x : x \in A \Rightarrow x \in B$ (А - подмножество B)\\
$A = B$, то есть $A \subset B \wedge B \subset A$ (A равно B)\\
$A \subsetneq B$, то есть $A \subset B \wedge A \ne \varnothing \wedge A \ne B$ (A - собственное подмножество B)\\
Способы задать множетсво:
\begin{itemize}
    \item Полное задание: $\{a,b,c\}$.
    \item Неполное: $a_1, a_2, ..., a_k$. Но должно быть понятно как образована последовательно. Например $\{1,5,...,22\}$ — непонятно.
    \item Можно так же и бесконечные: $\{a1, a2, ...\}$.
    \item Словесным описанием. Например, множество простых чисел.
    \item Формулой. Например, пусть задана функция $F(x)$ — функция для всех чисел, которая возращает истину или ложь. Тогда можно взять множество $\{x : F(x)\}$.
\end{itemize}
Операции с множествами:\\
\begin{center}
  \renewcommand{\arraystretch}{1.7}
  \begin{tabular}{| c | c | c |}
    \hline
    \textbf{Символ} & \textbf{Определение} & \textbf{Описание}\\
    \hline
    {\Large $\cap$} & $A \cap B = \{ x \mid x \in A \land x \in B\}$ & Пересечение множеств\\
    \hline
    {\Large $\bigcap_{k=1}^n A_k$} & $A = A_1 \cap A_2 \cap \ldots \cap A_n$ & Пересечение множества множеств \\
    \hline
    {\Large $\cup$} & $A \cup B = \{ x \mid x \in A \lor x \in B\}$ & Объединение множеств\\
    \hline
    {\Large $\bigcup_{k=1}^n A_k$} & $A = A_1 \cup A_2 \cup \ldots \cup A_n$ & Объединение множества множеств \\
    \hline
    {\Large $\setminus$} & $A \setminus B = \{ x \mid x \in A \land x \notin B\}$ & Разность множеств\\
    \hline
    {\Large $\times$} & $A \times B = \{ (x,\,y) \mid x \in A, y \in B\}$ & Декартово произведение\\
    \hline
    {\Large $\bigtriangleup$} & $A \bigtriangleup B = (A \setminus B) \cup (B \setminus A)$ & Симметрическая разность\\
    \hline
    {\Large $\varnothing$} & $\forall x: \; x \notin \varnothing$ & пустое множество\\ 
    \hline
    {\Large $\mathbb{N}$} & & Натуральные числа\\
    \hline
    {\Large $\mathbb{Z}$} & & целые числа \\
    \hline
    {\Large $\mathbb{Q}$} & $\frac{a}{b}$, где $a \in \mathbb{Z}, b \in \mathbb{N}$ & рациональные числа \\
    \hline
    {\Large $\mathbb{R}$} & & действительные числа \\
    \hline
    {\Large $2^X$} & & множество всех подмножеств $X$ \\
    \hline
  \end{tabular}
\end{center}
Важный момент: $1 \in \{1\}$, но  $1 \notin\{\{1\}\}$\\
Правила де Моргана:\\
Пусть есть $A_\alpha \subset X$
\begin{enumerate}
     \item $X \setminus \bigcup_{\alpha \in I} A_\alpha = \bigcap_{\alpha \in I} X \setminus A_\alpha$.
     \item $X \setminus \bigcap_{\alpha \in I} A_\alpha = \bigcup_{\alpha \in I} X \setminus A_\alpha$.
\end{enumerate}
Доказательство: $X \setminus \bigcup_{\alpha \in I} A_{\alpha} = \{x: x \in X \land x \notin A_{\alpha} \; \forall \alpha \in I\} = \{x: \forall \alpha \in I X \setminus A_{\alpha}\} = \bigcap_{\alpha \in I} X \setminus A_{\alpha}$.\\
Упорядоченная пара $\left<x,y\right>$. Важное свойство $\left<x, y\right> = \left<x',y'\right> \iff x = x' \land y = y'$\\
\\
\\

2. \textbf{Отношения: область определения, область значений, обратное отношение, композиция отношений, свойства, примеры.}\\
Отношение $R \subset X \times Y$. $x$ и $y$ находятся в отношении $R$, если их $\left<x, y\right> \in R$.
\begin{itemize}
    \item Область определения $\delta_R = {dom}_R = \{x \in X: \; \exists y \in Y: \; \left<x, y\right> \in R$.
    \item Область значений $\rho_R = {ran}_R = \{y \in Y: \; \exists x \in X: \; \left<x, y\right> \in R$
    \item Обратное отношение $R^{-1} \subset Y \times X \; \; R^{-1} = \{\left<x,y\right>\} \in R$.
    \item Композиция отношения. $R_1 \subset X \times Y, R_2 \subset Y \times Z: \; R_1 \circ R_2 \subset X \times Z$.
    \item $R_1 \circ R_2 = \{\left<x, z\right> \in X \times Z\; \vert \; \exists y \in Y: \; \left<x,y\right> \in R_1 \land \left<y,z\right> \in R_2\}$
\end{itemize}

Свойства:
\begin{enumurate}
    \item Функция из $X$ в  $Y$ --- отношение ($\delta_f = X$), для которого верно:
    \[
    \left. \begin{array}{l} \left<x,y\right> \in f \\ \left<x, z\right> \in f \end{array} \right\} \Rightarrow y = z
    .\]
    Используется запись $y = f(x)$. 
    \item Последовательность - функция у которой $\delta_f = \mathbb{N}$
    \item Отношение $R$ называется рефлективным, если $\forall x: \; \left<x, x\right> \in R$.
    \item Отношение $R$ называется симметричным, если  $\forall x, y \in X: \; \left<x, y\right> \in R \Rightarrow \left<y, x\right> \in R$
    \item Отношение $R$ называется иррефлективным, если  $\forall x \left<x,x\right> \notin R$
    \item Отношение $R$ называется антисимметричным, если  $\left. \begin{array}{r} \left<x, y\right> \in R \\ \left<y, x\right> \in R\end{array} \right\} \Rightarrow x = y$
    \item Отношение $R$ называется транзитивным, если  $\left. \begin{array}{r} \left<x, y\right> \in R \\ \left<y, z\right> \in R\end{array} \right\} \Rightarrow \left<x, z\right> \in R$\\
    \item Отношение называется отношением эквивалентности, если отношение рефлективно, симметрично, транзитивно. Например: Равенство, сравнение по модулю $\mathbb{Z}$,  $\|$, отношение подобия треугольников.
    \item Если выполняется рефлективность, антисимметричность и транзитивность, от данное отношение --- отношение нестрогого частичного порядка. Например: $\ge$; $A \subset B$ на $2^X$.
    \item Если выполняется иррефлективность и транзитивность, то данное отношение --- отношение строгого частичного порядка. Например: $>$;  $A$ собственное подмножество  $B$ на  $2^X$.
    \item $R$ - нестрогий ч.п.  $\Rightarrow$ $R = \{\left<x,y\right> \in R: \; x \neq y\}$ --- строгий ч.п.
\end{enumurate}

Примеры отношений:
\begin{itemize}
    \item Отношение равенства. $R = \{\left<x,x\right>: \; x \in X\}$. Но это просто равенство.
    \item "$\ge$" ($X = \mathbb{R}$). $R = \{ \left<x,y\right>: \; x \ge y\}$
    \item "$>$" ($X = \mathbb{R}$). $R = \{\left<x,y\right>: x > y\}$ \\
        $\delta_{>} = {2,3,4\ldots}$\\
        $\rho_> = \mathbb{N}$\\
        $>^{-1} = < = \{\left<x,y\right>: \; x < t\}$ \\
        $> \circ > = \{\left<x,z\right>\; x-z\ge2\}$
    \item $X$ --- прямые на плоскости. "$\perp$":  $R = \{\left<x,y\right>: \; x \perp y\}$. \\
            $\delta_\perp = \rho_\perp = X$ \\
            $\perp^{-1} = \perp$\\
            $\perp \circ \perp = \|$
    \item $\left<x, y\right> \subset R$, когда  $x$ --- отец  $y$. \\ 
        $\delta_R = \{\text{Все, у кого есть сыновья}\}$. \\
        $\rho_R$ --- религиозный вопрос. См. Библию \\
        $R^{-1} = \text{сын}$ \\
        $R \circ R = \{\text{дед по отцовской линии}\}$
\end{itemize}

3. \textbf{Аксиомы вещественных чисел. Математическая индукция. Существование наибольшего и наименьшего элемента в конечном множестве. Следствия.}\\
Есть две операции:
\begin{itemize}
    \item $+: \mathbb{R} \times \mathbb{R} \to \mathbb{R}$.
        \begin{itemize}
            \item Коммутативность. $x+y=y+x$.
            \item Ассоциативность.  $(x+y)+z=x+(y+z)$
            \item Существует ноль.  $\exists 0 \in \mathbb{R} \; \; x + 0 = x$
            \item Существует противоположный элемент. $\exists (-x) \in \mathbb{R} \; \; x+(-x) = 0$
        \end{itemize}
    \item $\cdot: \mathbb{R} \times \mathbb{R} \to \mathbb{R}$.
        \begin{itemize}
            \item Коммутативность. $x\cdot y=y\cdot x$.
            \item Ассоциативность.  $(x\cdot y)\cdot z=x\cdot (y\cdot z)$
            \item Существует единица.  $\exists 1 \in \mathbb{R} \; \; x \cdot 1 = x$
            \item Существует обратный элемент. $\exists x^{-1} \in \mathbb{R} \; \; x \cdot x^{-1} = 1$
        \end{itemize}
\end{itemize}
Свойство дистрибутивности: $(x+y) \cdot z = x \cdot z + y \cdot z$. Структура с данными операциями называется полем.

Введем отношение $\le$: Оно рефлексивно, антисимметрично и транизитивно, то есть нестрогий частичного порядка:
\begin{enumerate}
    \item $a \le a$
    \item $a \le b \wedge b \le a \Rightarrow a = b$
    \item $a \le b \wedge b \le c \Rightarrow a \le c$
    \item $\forall a, b$ выполняется $a \le b \vee b \le a$
    \item С операцией + : $\forall c$ выполняется $a \le b \Rightarrow a + c \le b + c$
    \item С операцией $\cdot$ : $a > 0, b > 0 \Rightarrow ab > 0$
\end{enumerate}
Аксиома полноты: Если $A$ и $B \subset \mathbb{R}$ и $\forall a \in A, b \in B: \; a \le b$ и $A \neq \varnothing \land B \neq \varnothing$, тогда $\exists c \in \mathbb{R}\; a \le c \le b$ (+ $A$ левее $B$).\\
Множество рациональных не удовлетворяет аксиоме полноты.\\
Например: $A = \{x \in \mathbb{Q} \; \vert \; x^2 < 2\}$,  $B = \{x \in \mathbb{Q} \; \vert \; x>0 \land x^2 > 2\}$. Единственная точка, между этими множествами --- $\sqrt{2}$ 

Пусть $P_n$ - последовательность утверждений. Тогда, если  $P_1$ --- верное и из того, что $P_n$ --- верно следует, что  $P_{n+1}$ --- верно. Тогда все $P_n$ верны  $\forall n \in \mathbb{N}$

Наибольшие/наименьшие элементы:
В непустом конечном множестве $A$ есть наибольший и наименьший элементы.\\
Доказательство. Докажем по индукции:
\begin{itemize}
    \item База. $|A| = 1$. Очевидно.
    \item Переход.  $n \to n + 1$.
    \item Доказательство. Рассмотрим множество из  $n + 1$ элемента  $\{x_1\ldots x_n,x_{n+1}\}$. Выкинем из него последний элемент. Тогда по индукционному предположению у нас есть максимальный элемент $x_k$. Тогда рассмотрим два случая:
         \begin{enumerate}
             \item $x_k \ge x_{n+1}$. Тогда $x_k$ --- наибольший элемент множества $\{x_1\ldots x_n,x_{n+1}\}$.
             \item $x_k < x_{n+1}$. Тогда по транзитивности  $x_{n+1}$ больше всех других элементов множества. Значит, $x_{n+1}$ --- наибольший элемент множества $\{x_1\ldots x_n,x_{n+1}\}$.
        \end{enumerate}
\end{itemize}

4. \textbf{Принцип Архимеда. Следствия. Наибольший элемент в множестве целых чисел. Существование целой части числа.}\\
Принцип Архимеда:\\
Пусть $x \in \mathbb{R} \land y > 0$. Тогда  $\exists n \in \mathbb{N}:\; x < ny$\\
Доказательство:\\
Фиксируем $g$\\
$A = \{a \in \mathbb{R}$ | $a < yn\}$\\
$B \mathbb{R}\setminus A \Rightarrow B = \{b \in \mathbb{R}$ | $\forall n : b \ge yn\}$\\
Пусть $B$ левее $A$ : $\forall a, b; \exists n$ : \begin{cases}
    $b \le a$
    \\
    $a < yn$
\end{cases} $\Rightarrow b < yn ??? \Rightarrow A$ левее $B \Rightarrow \exists c$ : \begin{cases}
    $\forall \le c$
    \\
    $\forall \ge c$
\end{cases} $\Rightarrow$ \begin{cases}
    $c - y = a' \in A$
    \\
    $c + y = b' \in B$
\end{cases} $\Rightarrow c = a' + y < yn' + y = y(n' + 1) = y\widetilde{n}$\\
$b' = c + y \le y\widetilde{n} + y = y(\widetilde{n} + 1) = yn \Rightarrow b' \le yn ??? \Rightarrow$ \begin{cases}
    $B = \varnothing \Rightarrow A = \mathbb{R}$\\
    $A = \mathbb{R}\setminus B$
\end{cases}

Следствие:\\
Если $\epsilon > 0$, то $\exists n \in \mathbb{N} \; \frac{1}{n} < \epsilon$\\
Доказательство:\\
$x = 1, y = \epsilon \Rightarrow ny = n \epsilon > x = 1 \iff \epsilon > \frac{1}{n}$\\

Теорема:\\
В непустом ограниченном сверху (снизу) множестве целых чисел есть наибольший (наименьший) элемент.\\
Доказательство:\\
Пусть $A \subset \mathbb{Z}$.  $c$ --- его верхняя граница.\\
Возьмем  $b \in A$ и рассмотрим  $B := {x \in A \mid x \ge b}$. Заметим, что $B$ содержит конечное число элементов, значит в нем есть наибольший элемент. Пусть это $m \in B$:  $\forall x \in B:\; x \le m$. Докажем, что $m$ --- наибольший элемент и в  $A$.\\
Для этого заметим, что любой $x \in A$ либо лежит в $B$, либо  $x < b$, а по транзитивности  $x < b \le m$.\\

Пусть $x \in \mathbb{R}$, тогда  $[x] = \lfloor x \rfloor$ --- наименьшее целое число, не превосходящее  $x$.
\begin{enumerate}
    \item $[x]  \le x < [x] + 1$\\
        Левое неравенство очевидно. Правое неравенство можно доказать от противного: пусть $x \ge [x] + 1$, тогда справа целое число большое $[x]$, но меньшее  $x$. Противоречие.
    \item $x - 1 < [x] \le x$
\end{enumerate}

5. \textbf{! Супремум и инфимум. Определение и теорема существования. Характеристика супремума.}\\
Определение. Пусть $A \subset \mathbb{R}$. Тогда  $A$ --- ограничено сверху, если  $\exists c \in \mathbb{R}: \; \forall a \in A\; a \le c$. Такое $c$ называется верхней границей.\\

Определение. Пусть $A \subset \mathbb{R}$. Тогда  $A$ --- ограничено снизу, если  $\exists b \in \mathbb{R}: \; \forall a \in A\; a \ge b$. Такое  $b$ называется нижней границей.\\

Определение. Пусть $A \subset \mathbb{R}$. Тогда  $A$ --- ограничено, если оно ограничено сверху и снизу. Например: $\mathbb{N}$ не ограничено сверху, но ограничено снизу.\\

Доказательсво. Пусть $\exists c \in \mathbb{R}: \; c \ge n\; \forall n \in \mathbb{N}$. Тогда это противоречит принципу Архимеда при $x = c, y = 1$. \\ Для ограниченности снизу достаточно взять $c=-1$.\\

$\sup A$ --- наименьшая верхняя граница множества $A$\\
$A \subset \mathbb{R}$, $a \ne \varnothing$, $A$ - ограничено сверху\\
Характеристика: если $A \subset \mathbb{R}$, $A \ne \varnothing$, $A$ - ограничено сверху, тогда $\exists\sup A$\\

$\inf A$ --- наибольшая нижняя граница множества $A$\\
$A \subset \mathbb{R}$, $a \ne \varnothing$, $A$ - ограничено снизу\\

Доказательсво существования: $B = \{\text{все верхние границы множества $A$}\}$\\
$B \subset \mathbb{R}$, $B \ne \varnothing$, $A \text{ левее } B$, $a \le b\  \forall a \in A$, $\forall b \in B \Rightarrow \exists c \in \mathbb{R}$ : \begin{cases}
    $a \le c$ - верх $A$
    \\
    $c \le b$ - низ $A$
\end{cases} $\Rightarrow c = \sup A$\\
Аналогично с $\inf A$\\

6. \textbf{! Теорема о вложенных отрезках. Существенность условий.}\\
Теорема о вложенных отрезках:\\
$[a_1; b_1] \supset [a_2; b_2] \supset ... \supset [a_n; b_n] \supset ...$\\
Тогда $\bigcap_{n=1}^{+\infty} [a_n; b_n] \ne \varnothing\  (\exists c\ : \ \forall n\  c \in [a_n; b_n])$\\
Доказательство:\\
\begin{cases}
    $A = \{a_1, a_2, ..., a_n, ...\}$, $A \ne \varnothing$\\
    $B = \{b_1, b_2, ..., b_n, ...\}$, $B \ne \varnothing$\\
    $A\text{ левее }B$,\ $a_k \le b_m \ \forall k,m$
\end{cases}$\Rightarrow$\\$\Rightarrow$ по аксиоме полноты $\exists c$,\ $\forall a \in A$, $\forall b \in B$ : $a \le c \le b \Rightarrow \forall n\  a_n \le c \le b_n \Rightarrow c \in [a_n; b_n] \forall n \in \mathbb{N} \Rightarrow \bigcap_{n=1}^{+\infty} [a_n; b_n] \ne \varnothing$\\
$A\text{ левее }B$, так как:
\begin{enumerate}
    \item $k = m$
    \item $m > k \ a_k \le a_m \le b_n$
    \item $m < k \ a_k \le b_k \le b_m$
\end{enumerate}
Существенность условий: для лучей и полуинтервалов неверно.\\

7. \textbf{! Монотонные и ограниченные последовательности. Два определения предела и их равносильность. Примеры.}\\
$\{x_n\}_{n=1}^{+\infty}$ - возрастающая (неубывающая), если $\forall n\ x_{n+1} \ge x_n$\\
$\{x_n\}_{n=1}^{+\infty}$ - строго возрастрает, если $\forall n\ x_{n+1} > x_n$\\
$\{x_n\}_{n=1}^{+\infty}$ - убывающая, если $\forall n\ x_{n+1} \le x_n$\\
$\{x_n\}_{n=1}^{+\infty}$ - строго убывающая, если $\forall n\ x_{n+1} < x_n$\\
$\{x_n\}_{n=1}^{+\infty}$ - ограниченна сверху, если $\exists m \in \mathbb{R}:\ \forall n\ x_n \le m$\\
$\{x_n\}_{n=1}^{+\infty}$ - ограниченна снизу, если $\exists m \in \mathbb{R}:\ \forall n\ x_n \ ge m$\\
$\{x_n\}_{n=1}^{+\infty}$ - ограниченная, если ограниченна сверху и снизу\\
Опеределения предела:\\
\textit{Не classic}\\
$a = \lim_{n \rightarrow +\infty}x_n$, если  все любого интервала, содержащего точку $a$, находиться лишь конечное число членов.\\
\textit{Classic}\\
$a = \lim_{n \rightarrow +\infty}x_n \Leftrightarrow \forall \epsilon > 0\ \exists N \in \mathbb{N}:\ \forall n \ge N\ a - \epsilon < x_n < a + e$\\
$|x_n - a| < \epsilon$\\
Равносильность состоит в том, что внутри отрезка $[a-\epsilon; a+\epsilon]$ у нас находится бесконечное число членов, а значит вне его находится конечное число членов.\\

8. \textbf{! Простейшие свойства пределов последовательностей (единственность предела, предельный переход в неравенстве, ограниченность).}
Единственность предела:\\
Если существует два разных предала одной последовательности, то вне двух интервалах находится конечное число членов $\Rightarrow\{x_n\}_{n=1}^{+\infty}$ имеет конечное число членов???\\

Ограниченность:\\
Если последовательность имеет предел, то она ограничена\\
$\exists a:\ \lim_{n \rightarrow +\infty} = a \Rightarrow \{x_n\}_{n=1}^{+\infty}$\\
Рассмотрим интервал $[a-1; a+1]$\\
max(($a+1$), наибольшее невошедшее) = верхняя граница\\
min(($a-1$), наименьшее невошедшее) = нижняя граница\\
Из ограниченности НЕ следует существование предела (Например: $x_n = (-1)^n$)\\

Если изменить конечное число членов последовательности, то предел не изменится или не появится. Если добавить/удалить/переставить конечное число элементов, то не изменится и не появится.\\

Предельный переход в неравенстве:\\
\begin{cases}
    $\forall n\ x_n \le y_n$\\
    \begin{cases}
        \lim x_n = a\\
        \lim y_n = b
    \end{cases}
\end{cases} $\Rightarrow a \le b$\\
Доказательство:\\
Пусть $a > b$\\
$\lim x_n = a \Rightarrow \forall \epsilon_1 > 0\ \exists N_1:\ \forall n_1 > N_1\ |x_n - a| < \epsilon_1$\\
$\lim y_n = b \Rightarrow \forall \epsilon_2 > 0\ \exists N_2:\ \forall n_2 > N_2\ |y_n - b| < \epsilon_2$\\
$\forall\epsilon\ N = \max(N_1, N_2):\ \forall n > N$ \begin{cases}
    $|x_n - a| < \epsilon$\\
    $|y_n - b| < \epsilon$
\end{cases} $\Rightarrow$ \begin{cases}
    $a - \epsilon < x_n < a + \epsilon$\\
    $b - \epsilon < y_n < b + \epsilon$\\
    $a > b \Rightarrow \exists\epsilon:\ b + \epsilon < a - \epsilon$
\end{cases} $\Rightarrow y_n < x_n??? \Rightarrow a \le b$

9. \textbf{! Теорема о стабилизации знака и теорема о двух милиционерах. Следствия.}\\
Теорема о стабилизации знака и теорема:\\
$\forall n\ x_n < y_n \Rightarrow a \le b$\\
Следствия:\\
$\lim x_n = a$\\
$\forall n:\ x_n \ge A \Rightarrow a \ge A$\\
$\forall n:\ x_n \le B \Rightarrow a \le B$\\
$\forall n:\ x_n \in [\alpha; \beta] \Rightarrow a \in [\alpha; \beta]$\\
Теорема о двух миллиционерах:\\
$\forall n \in \mathbb{N}:$ \begin{cases}
    $\lim x_n = a$\\
    $\lim z_n = a$\\
    $x_n \le y_n \le z_n$
\end{cases} $\Rightarrow \lim y_n = a$\\
Доказательство:\\
Фиксируем $\epsilon > 0$\\
$\exists N_1:\ \forall n \ge N_1\ |x_n - a| < \epsilon$\\
$\exists N_2:\ \forall n \ge N_2\ |z_n - a| < \epsilon$\\
$N = \max(N_1, N_2)$\\
$\forall n \ge N$ \begin{cases}
    $a - \epsilon < x_n < a + \epsilon$\\
    $a - \epsilon < y_n < a + \epsilon$
\end{cases} $\Rightarrow a - \epsilon < x_n \le y_n \le z_n < a + \epsilon \Rightarrow a - \epsilon < y_n < a + \epsilon \Rightarrow |y_n - a| < \epsilon \Rightarrow \lim y_n = a$\\

10. \textbf{! Предел монотонной последовательности.}\\
Теорема о пределе монотонной последовательности:\\
$\{x\}_{x=1}^{+\infty}$ - возрастает и ограниченна сверху, тогда $\exists \lim x_n = S = \sup\{x_1, ...\}$\\
$\{y\}_{x=1}^{+\infty}$ - убывает и ограниченна снизу, тогда $\exists \lim y_n$\\
$\{z\}_{x=1}^{+\infty}$ - монотонная (+ ограниченная?), тогда $\exists \lim z_n \Leftrightarrow \{z_n\}$ - ограниченная\\
Доказательство:\\
$\{x\}_{n=1}^{+\infty}$ - ограниченная сверху $\Rightarrow \exists \sup\{x_1, x_2, ..., x_n, ...\} = S = \sup x_n$\\
$\forall\epsilon > 0\ \exists x_{\widetilde{n}}:$ \begin{cases}
    $x_{\widetilde{n}} > S - \epsilon$\\
    ${x_n}$ - возрастает
\end{cases} $\Rightarrow \forall n \ge \widetilde{n}$\\
$S + \epsilon > S \ge x_n \ge x_{\widetilde{n}} > S - \epsilon \Rightarrow |x_n - S| < \epsilon \Rightarrow \lim x_n = S$\\

11. \textbf{Арифметические свойства пределов последовательности.}
\begin{enumerate}
    \item $\lim(x_n + y_n) = a + b$
    \item $\lim(x_n \cdot y_n = a \cdot b)$
    \item $\lim(c \cdot x_n) = c \cdot a$
    \item $\lim(\dfrac{x_n}{y_n}) = \dfrac{a}{b}$
    \item $\lim(|x_n + \alpha_n|) = |a|$, где $\alpha_n$ - б/м (то есть $\lim \alpha_n = 0$)
\end{enumerate}
Доказательства:\\
1. $\lim x_n \Leftrightarrow \lim(x_n + \alpha_n)$\\
$\lim y_n \Leftrightarrow \lim(y_n + \beta_n)$\\
$x_n + y_n = (x_n + \alpha_n) + (y_n + \beta_n) = (a + b) + (\alpha_n + \beta_n) = a + b + \gamma_n \Rightarrow \lim(x_n + y_n) = a + b$\\

2. \begin{cases}
    $x_n = a + \alpha_n$\\
    $y_n = b + \beta_n$
\end{cases} $\Rightarrow x_y \cdot y_n = (a + \alpha_n)\cdot(b + \beta_n) = ab + a\beta_n + \alpha_n b + \alpha_n\beta_n$ ($a\beta_n,\ \alpha_n b,\ \alpha_n\beta_n$) - б/м $\Rightarrow \lim(x_n\cdot y_n) = a\cdot b$\\

4. $\lim(\dfrac{x_n}{y_n}) = \lim(x_n \cdot \dfrac{1}{y_n})$\\
$\lim(\dfrac{1}{y_n})$, $y_n = b + \beta_n$\\
$\dfrac{1}{y_n} = \dfrac{1}{b + \beta_n} = \dfrac{1}{b} - \dfrac{1}{b} + \dfrac{1}{b + \beta_n} = \dfrac{1}{b} - \dfrac{b + \beta_n - b}{b(b + \beta_n)} = \dfrac{1}{b} - \beta_n \cdot \dfrac{1}{b(b + \beta_n)}$\\
$\dfrac{1}{b(b + \beta_n)}$ - ограниченная, так как $\lim \dfrac{1}{b(b + \beta_n)} = \dfrac{1}{b^2} \Rightarrow \lim(\dfrac{1}{y_b}) = \dfrac{1}{b} \Rightarrow \lim(\dfrac{x_n}{y_n}) = \lim(x_n \cdot \dfrac{1}{y_n}) = a \cdot \dfrac{1}{b} = \dfrac{a}{b}$\\

12. \textbf{! Бесконечные пределы. Бесконечно большие. Связь между бесконечно малыми и бесконечно большими. Аналоги теорем для бесконечных пределов.}\\
Вспомним сначала бесконечно малые:\\
$\{\alpha\}_{n+1}^{+\infty}$ - бесконечно малая $\Leftrightarrow \lim \alpha_n = 0$\\
Наблюдение: $\lim x_n = a \Leftrightarrow \{x_n - a\}_{n=1}^{+\infty}$ - бесконечно малая $\Leftrightarrow x_n - a = \alpha_n \Leftrightarrow x_n = a + \alpha_n$\\
Доказательство: $\lim x_n = a \Leftrightarrow \forall \epsilon > 0\ \exists N:\ \forall n \ge N:\ |x_n - a| < \epsilon \Rightarrow |(x_n - a) - 0| < \epsilon \Rightarrow \lim \alpha_n = 0$\\
Свойства:
\begin{itemize}
    \item $\alpha_n, \beta_n$ - б/м $\Rightarrow \alpha_n + \beta_n$ - б/м\\
    Доказательство:\\
    $\forall \epsilon > 0\ \exists N_1:\ \forall n \ge N_1\ |\alpha_n| < \epsilon$\\
    $\forall \epsilon > 0\ \exists N_2:\ \forall n \ge N_2\ |\beta_n| < \epsilon$\\
    $N = \max(N_1, N_2)\ \forall n \ge N$\\
    $|\alpha_n + \beta_n| \le |\alpha_n| + |\beta_n| < \dfrac{\epsilon}{2} + \dfrac{\epsilon}{2} = \epsilon \Rightarrow \forall \epsilon > 0\ \exists N:\ \forall n \ge N\ |\alpha_n + \beta_n| < \epsilon \Leftrightarrow \lim(\alpha_n + \beta_n) = 0 \Rightarrow \alpha_n + \beta_n$ - б/м
    \item $\alpha_n$ - б/м; $a_n$ - ограниченная $\Rightarrow a_n \cdot \alpha_n$ - б/м\\
    Доказательство:\\
    $\{a_n\}$ - ограниченная $\Leftrightarrow \exists M > 0:\ |\alpha_n| < M\ \forall n \in N$\\
    Фиксируем $\epsilon > 0 \Rightarrow \exists N:\ \forall n \ge N\ |\alpha_n| < \dfrac{\epsilon}{M}$\\
    $|a_n\cdot\alpha_n| = |a_n| \cdot |\alpha_n| < M \cdot \dfrac{\epsilon}{M} = \epsilon \Rightarrow \alpha_n \cdot a_n$ - б/м
    \item \begin{cases}
        $\alpha_n, \beta_n$ - б/м\\
        $p, q \in \mathbb{R}$
    \end{cases} $\Rightarrow p \cdot \alpha_n + q \cdot \beta_n$ -  б/м
    \item \begin{cases}
        $\alpha_n, \beta_n$ - б/м\\
        $\beta_n$ - огр.
    \end{cases} $\Rightarrow \alpha_n \cdot \beta_n$ -  б/м
\end{itemize}
Бесконечно большие последовательности:\\
Говорят, что $\lim x_n = +\infty$, если $\forall M\ \exists N:\ \forall n \ge N\ x_n > M$\\
Другими словами $\lim x_n = +\infty$, если вне любого луча вида $[M, +\infty)$ лежит лишь конечное число элементов последовательности\\

Говорят, что $\lim x_n = -\infty$, если $\forall M\ \exists N:\ \forall n \ge N\ x_n < M$\\
Другими словами $\lim x_n = -\infty$, если вне любого луча вида $(-\infty, M]$ лежит лишь конечное число элементов последовательности\\

Говорят, что $\lim x_n = \infty$, если $\forall M\ \exists N:\ \forall n \ge N\ |x_n| > M$\\
Другими словами $\lim x_n = \infty$, если вне любого луча вида $[-M, M]$ лежит лишь конечное число элементов последовательности\\

$\lim x_n = +\infty \Rightarrow \lim x_n = \infty$\\
$\lim x_n = -\infty \Rightarrow \lim x_n = \infty$\\

$\{x_n\}$ - б/б, если $\lim x_n = \infty$\\
Связь б/м и б/б:\\
$\forall n\ x_n \ne 0$, тогда $x_n$ - б/б $\Leftrightarrow \dfrac{1}{x_n}$ - б/м\\
Доказательство:\\
$x_n$ - б/б $\Leftrightarrow \forall M > 0:\ \exists N:\ \forall n \ge N\ |x_n| < M (> 0) \Leftrightarrow \dfrac{1}{|x_n|} < \dfrac{1}{M}$\\
$\dfrac{1}{M} \in [0; +\infty] = \epsilon$\\
$\forall \epsilon\ \exists N\ \forall n \ge N:\ \dfrac{1}{x_n} < \epsilon \Rightarrow \dfrac{1}{x_n}$ - б/м\\

Аналоги теорем для бесконечных пределов:
\begin{itemize}
    \item Предел единственный
    \item Стабилизация знака
    \item Предельный переход
    \item Теорема о двух гаишниках
\end{itemize}

13. \textbf{Арифметические действия в $\overline{\mathbb{R}}$. Примеры.}\\
$\overline{\mathbb{R}} = \mathbb{R}\cup\{-\infty\}\cup\{+\infty\}$

14. \textbf{Неравенство Бернулли. Предел $\lim a^n$}:\\
Неравенство Бернулли:\\
$x > -1,\ n \in \mathbb{N}$\\
$(1 + x)^n \ge 1 + nx$, причём true при $n = 1$ или $x = 0$\\
Доказательство (по ММИ):\\
База: при $n = 1$ очев true\\
Переход: пусть при $n$ true, тогда $(1 + x)^{n+1} \ge (1 + nx)(1 + x) = 1 + x + nx + nx^2 = 1 + (n + 1)x + nx^2$\\
Очев $nx^2 > 0 \Rightarrow 1 + (n + 1)x + nx^2 \ge 1 + (n + 1)x$\\

Предел $\lim a^n$: $q \in \mathbb{R}$
\begin{itemize}
    \item $|q| > 1 \Rightarrow \lim q^n = \infty$
    \item $|q| < 1 \Rightarrow \lim q^n = 0$
\end{itemize}
Доказательство: $|q| > 1 \Rightarrow |q| = 1 + x,\ x > 0$\\
$|q^n| = (1 + x)^n \ge 1 + nx > nx$\\
$\lim q^n = \infty$\\
$|q| < 1 \Leftrightarrow |\dfrac{1}{q}| > 1 \Rightarrow \dfrac{1}{q^n}$ - б/б $\Leftrightarrow q^n$ - б/м\\
$a = 0$ - очев\\

15. \textbf{! Определение экспоненты и числа e.}\\
Вывод экспоненты довольно душный, почитайте полную версию у вас в конспекте или попросите у однокрурсников. Вот краткая версия:\\
$x_n = (1 + \dfrac{a}{n})^n$ возрастает при $n > -a$, причём строго при $a \ne 0$ ($a \in \mathbb{R}$)\\
Доказательство:\\
$\dfrac{x_n}{x_{n-1}} = \dfrac{(1 + \dfrac{a}{n})^n}{(1 + \dfrac{a}{n-1})^n-1} =$ ................... $\ge \dfrac{n - 1}{n - 1 + a} \cdot \dfrac{n - 1 + a}{n-1} \ge 1 \Rightarrow n > -a$ и $x_n$ возрастает\\

$x_n = (1 + \dfrac{a}{n})^n$ - орг. сверху\\
Доказательство: $y_n = (1 + \dfrac{-a}{n})^n$\\
$x_n \cdot y_n = (1 + \dfrac{a}{n})^n \cdot (1 + \dfrac{-a}{n})^n = (1 - \dfrac{a^2}{n^2})^n \le 1$\\
$x_n \le \dfrac{1}{y_n} \le \dfrac{1}{y_{n-1}} \le \dfrac{1}{y_{n-2}} \le ... \le \dfrac{1}{y_1}\ (n > 0)$\\
$x_n \le \dfrac{1}{y_n} \le ... \le \dfrac{1}{y_{[a]+1}}\ (a > 0)$\\

Следствие:
\begin{enumerate}
    \item $\{x_n\}$ - имеет предел
    \item $z_n = (1 + \dfrac{1}{b})^n$ - убывает
\end{enumerate}
Доказательство: $x_n$ НСНМ возрастает, $x_n$ - огр. сверху $\Rightarrow$ теорема о пределе последовательности $\Rightarrow \exists \lim x_n \in \mathbb{R}$\\

Итак, имеем:\\
$\lin x_n = \lim(1 + \dfrac{a}{n})^n = \exp(a)$\\
$\lim(1 + \dfrac{1}{n})^n = e = \exp(1)$

16. \textbf{Свойства экспоненты.}\\
\begin{enumerate}
    \item $\exp(1) = e$\\
    $\exp(0) = 1$
    \item $a \le b \Rightarrow \exp(a) \le \exp(b)$\\
    $(1 + \dfrac{a}{n})^n \le (1 + \dfrac{b}{n})^n$\\
    Устремляем в $+\infty$\\
    $\exp(a) \le \exp(b)$
    \item $\exp(a) \ge 1 + a$\\
    $(1 + \dfrac{a}{n})^n \ge 1 + n \cdot \dfrac{a}{n} = 1 + a$\\
    Устремляем в $+\infty$\\
    $\exp(a) \ge 1 + a$
    \item $(\exp(a))(\exp(-a)) \le 1$
    \item $\exp(a) \le \dfrac{1}{1 - a}$\\
    $a < 1$\\
    $\exp(a) \cdot \exp(-a) \le 1$\\
    $\exp(a) \le \dfrac{1}{\exp(-a)}$
    \item $\forall n\ (1 + \dfrac{1}{n})^n < \epsilon$\\
    $\forall n\ (1 + \dfrac{1}{n})^{n+1} > \epsilon$\\
    const $n$, $k > n + 1$\\
    $(1 + \dfrac{1}{n})^n < (1 + \dfrac{1}{n})^{n+1} < ... < (1 + \dfrac{1}{k})^k$\\
    Устремляем в $+\infty$\\
    $e < e < ... < e$
    \item $2 < e < 3$
\end{enumerate}

17. \textbf{Формула для экспоненты суммы (с леммой).}\\
$\{a_n\}:\ \lim a_n = a$, тогда $\lim (1 + \dfrac{a_n}{n})^n = \exp(a)$\\
Док-во: $x_n = (1 + \dfrac{a_n}{n})^n;\ y_n = (1 + \dfrac{a}{n})^n$\\
$|x_n - y_n| = |(1 + \dfrac{a_n}{n})^n - (1 + \dfrac{a}{n})^n| = |A^n - B^n|$
\begin{cases}
    $A = (1 + \dfrac{a_n}{n})^n > 0,\ \exists N_1\ \forall n > N_1$\\
    $B = (1 + \dfrac{a}{n})^n > 0,\ \exists N_2\ \forall n > N_2$
\end{cases} $\bigcap$ \begin{cases}
    $A < 1 + \dfrac{M}{n}$\\
    $B < 1 + \dfrac{M}{n}$\\
\end{cases} $\exists M > 0$
$|A^n - B^n| = |A - B|\cdot(\sum_{i=0}^n A^{n-i}\cdot B^{i-1}) < |A - B|\cdot(\sum_{i=0}^n (1 + \dfrac{M}{n})^{n-i}\cdot (1 + \dfrac{M}{n})^{i-1} = |A - B|\cdot(\sum_{i=0}^n(1 + \dfrac{M}{n})^{n-1}) = |A - B|\cdot(1 + \dfrac{M}{n})^{n-1}\cdot n = \dfrac{|a_n - a|}{n}\cdot(1 + \dfrac{M}{N})^{n-1}\cdot n = |a_n - a|(1 + \dfrac{M}{n})^n(1 + \dfrac{M}{n})^{-1}$\\
Устремим в бесконечность и получим: $0\cdot\exp(M)\cdot 1 \Rightarrow |a_n - a|(1 + \dfrac{M}{n})^n(1 + \dfrac{M}{n})^{-1} \longrightarrow 0 \Rightarrow \{|A^n - B^n|\} \longrightarrow 0 \Rightarrow {x_n - y_n}$ - б/м $\Rightarrow \lim x_n = \lim y_n = a$\\

$(1 + \dfrac{a}{n})^n\cdot(1 + \dfrac{b}{n})^n = (1 + \dfrac{b}{n} + \dfrac{a}{n} + \dfrac{ab}{n^2}) = (1 + \dfrac{a + b + \frac{ab}{n}}{n})$\\
Устремляем в бесконечность\\
$\exp(a)\cdot\exp(b) = \exp(a + b)$

18. \textbf{Сравнение скорости возрастания последовательностей $n^k$, $a^n$, $n!$ и $n^n$.}\\
$n^k < a^n < n! < n^n$\\
\begin{enumerate}
    \item $x_n = \dfrac{n^k}{a^n},\ \forall n:\ x_n > 0$ (так как $n > 0, a > 1$)\\
    $\lim(\dfrac{x_{n+1}}{x_n}) = \lim(\dfrac{(n+1)^ka^n}{a^{n+1}n^k}) = \lim(\dfrac{1}{a}\cdot\dfrac{(n+1)^k}{n^k}) =$ \begin{cases}
        $x_n > 0$\\
        $\dfrac{1}{a} < 1$
    \end{cases} $\Rightarrow \lim(\dfrac{n^k}{a^n}) = 0 \Rightarrow n^k < a^n$
    \item $x_n = \dfrac{a^n}{n!},\ \forall n:\ x_n > 0$ (так как $n > 0, a > 1$)\\
    $\lim(\dfrac{x_{n+1}}{x_n}) = \lim(\dfrac{a^{n+1}n!}{a^n(n+1)!}) = \lim(a\cdot\dfrac{1}{n+1}) =$ \begin{cases}
        $x_n > 0$\\
        $0 < 1$
    \end{cases} $\Rightarrow \lim(\dfrac{a^n}{n!}) = 0 \Rightarrow a^n < n!$
    \item $x_n = \dfrac{n!}{n^n},\ \forall n:\ x_n > 0$ ($n > 0$)\\
    $\lim(\dfrac{x_{n+1}}{x_n}) = \lim(\dfrac{(n+1)!n^n}{(n+1)^{n+1}n!}) = \lim(\dfrac{n!}{(n+1)^n}) = \lim(\dfrac{1}{(1 + \dfrac{1}{n})^n}) = $ \begin{cases}
        $x_n > 0$\\
        $\dfrac{1}{e} < 1$
    \end{cases} $\Rightarrow \lim(\dfrac{n!}{n^n}) = 0 \Rightarrow n! < n^n$
\end{enumerate}

19. \textbf{Теорема Штольца (для неопределенности $\frac{\infty}{\infty}$). Сумма m-ых степеней натуральных чисел.}

20. \textbf{Теорема Штольца (для неопределенности $\frac{0}{0}$).}

21. \textbf{Подпоследовательности (определение и простейшие свойства). Теорема о стягивающихся отрезках. }

22. \textbf{! Теорема Больцано–Вейерштрасса (в том числе и случай неограниченной последовательности).}

23. \textbf{! Фундаментальные последовательности. Свойства. Критерий Коши.}

24. \textbf{Верхний и нижний пределы. Частичные пределы. Связь между ними.}

25. \textbf{Характеристика верхних и нижних пределов с помощью $N$ и $\epsilon$. Сохранение неравенств для верхних и нижних пределов.}

26. \textbf{! Сходимость рядов. Необходимое условие сходимости рядов. Примеры.}

27. \textbf{Простейшие свойства сходящихся рядов.}

28. \textbf{Окрестности и проколотые окрестности. Предельные точки множества.}

29. \textbf{! Определения предела функций в точке. Простейшие свойства.}

30. \textbf{! Равносильность определения предела по Коши и по Гейне.}

31. \textbf{Свойства функций, имеющих предел.}

32. \textbf{Арифметические действия с пределами.}

33. \textbf{! Теорема о предельном переходе в неравенствах. Теорема о двух милиционерах.}

34. \textbf{! Критерий Коши для предела функций.}

35. \textbf{Левый и правый пределы. Предел монотонной функции.}

\end{document}
