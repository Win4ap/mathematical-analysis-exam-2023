\documentclass[12pt]{article}
\usepackage[utf8]{inputenc}
\usepackage[english,russian]{babel}
\usepackage{graphicx}
\usepackage{amsmath}
\usepackage{amsthm}
\usepackage{caption}
\usepackage{dsfont}
\usepackage{tikz}
\usepackage{amssymb}
\usepackage{subcaption}
\usepackage{imakeidx}
\usepackage{hyperref}
\usepackage[russian]{cleveref}
\usepackage[a4paper,left=15mm,right=15mm,top=30mm,bottom=20mm]{geometry}
\parindent=0mm
\parskip=3mm
\DeclareRobustCommand{\divby}{%
\mathrel{\text{\vbox{\baselineskip.65ex\lineskiplimit0pt\hbox{.}\hbox{.}\hbox{.}}}}%
}

\makeindex
\pagestyle{empty}
\title{Матанализ}
\author{Канта контроль}
\date{\today}
\begin{document}
\maketitle
\large

1. \textbf{Множества: упорядоченная пара, декартово произведение, операции над множествами. Правила де Моргана.}\\
Множество - какой-то набор элементов. Для любого элемента можно сказать принадлежит множеству или нет.\\
$A \subset B$, то есть $\forall x : x \in A \Rightarrow x \in B$ (А - подмножество B)\\
$A = B$, то есть $A \subset B \wedge B \subset A$ (A равно B)\\
$A \subsetneq B$, то есть $A \subset B \wedge A \ne \varnothing \wedge A \ne B$ (A - собственное подмножество B)\\
Способы задать множетсво:
\begin{itemize}
    \item Полное задание: $\{a,b,c\}$.
    \item Неполное: $a_1, a_2, ..., a_k$. Но должно быть понятно как образована последовательно. Например $\{1,5,...,22\}$ — непонятно.
    \item Можно так же и бесконечные: $\{a1, a2, ...\}$.
    \item Словесным описанием. Например, множество простых чисел.
    \item Формулой. Например, пусть задана функция $F(x)$ — функция для всех чисел, которая возращает истину или ложь. Тогда можно взять множество $\{x : F(x)\}$.
\end{itemize}
Операции с множествами:\\
\begin{center}
  \renewcommand{\arraystretch}{1.7}
  \begin{tabular}{| c | c | c |}
    \hline
    \textbf{Символ} & \textbf{Определение} & \textbf{Описание}\\
    \hline
    {\Large $\cap$} & $A \cap B = \{ x \mid x \in A \land x \in B\}$ & Пересечение множеств\\
    \hline
    {\Large $\bigcap_{k=1}^n A_k$} & $A = A_1 \cap A_2 \cap \ldots \cap A_n$ & Пересечение множества множеств \\
    \hline
    {\Large $\cup$} & $A \cup B = \{ x \mid x \in A \lor x \in B\}$ & Объединение множеств\\
    \hline
    {\Large $\bigcup_{k=1}^n A_k$} & $A = A_1 \cup A_2 \cup \ldots \cup A_n$ & Объединение множества множеств \\
    \hline
    {\Large $\setminus$} & $A \setminus B = \{ x \mid x \in A \land x \notin B\}$ & Разность множеств\\
    \hline
    {\Large $\times$} & $A \times B = \{ (x,\,y) \mid x \in A, y \in B\}$ & Декартово произведение\\
    \hline
    {\Large $\bigtriangleup$} & $A \bigtriangleup B = (A \setminus B) \cup (B \setminus A)$ & Симметрическая разность\\
    \hline
    {\Large $\varnothing$} & $\forall x: \; x \notin \varnothing$ & пустое множество\\ 
    \hline
    {\Large $\mathbb{N}$} & & Натуральные числа\\
    \hline
    {\Large $\mathbb{Z}$} & & целые числа \\
    \hline
    {\Large $\mathbb{Q}$} & $\frac{a}{b}$, где $a \in \mathbb{Z}, b \in \mathbb{N}$ & рациональные числа \\
    \hline
    {\Large $\mathbb{R}$} & & действительные числа \\
    \hline
    {\Large $2^X$} & & множество всех подмножеств $X$ \\
    \hline
  \end{tabular}
\end{center}
Важный момент: $1 \in \{1\}$, но  $1 \notin\{\{1\}\}$\\
Правила де Моргана:\\
Пусть есть $A_\alpha \subset X$
\begin{enumerate}
     \item $X \setminus \bigcup_{\alpha \in I} A_\alpha = \bigcap_{\alpha \in I} X \setminus A_\alpha$.
     \item $X \setminus \bigcap_{\alpha \in I} A_\alpha = \bigcup_{\alpha \in I} X \setminus A_\alpha$.
\end{enumerate}
Доказательство: $X \setminus \bigcup_{\alpha \in I} A_{\alpha} = \{x: x \in X \land x \notin A_{\alpha} \; \forall \alpha \in I\} = \{x: \forall \alpha \in I X \setminus A_{\alpha}\} = \bigcap_{\alpha \in I} X \setminus A_{\alpha}$.\\
Упорядоченная пара $\left<x,y\right>$. Важное свойство $\left<x, y\right> = \left<x',y'\right> \iff x = x' \land y = y'$\\
\\
\\

2. \textbf{Отношения: область определения, область значений, обратное отношение, композиция отношений, свойства, примеры.}

3. \textbf{Аксиомы вещественных чисел. Математическая индукция. Существование наибольшего и наименьшего элемента в конечном множестве. Следствия.}

4. \textbf{Принцип Архимеда. Следствия. Наибольший элемент в множестве целых чисел. Существование целой части числа.}

5. \textbf{! Супремум и инфимум. Определение и теорема существования. Характеристика супремума.}

6. \textbf{! Теорема о вложенных отрезках. Существенность условий.}

7. \textbf{! Монотонные и ограниченные последовательности. Два определения предела и их равносильность. Примеры.}

8. \textbf{! Простейшие свойства пределов последовательностей (единственность предела, предельный переход в неравенстве, ограниченность).}

9. \textbf{! Теорема о стабилизации знака и теорема о двух милиционерах. Следствия.}

10. \textbf{! Предел монотонной последовательности.}

11. \textbf{Арифметические свойства пределов последовательности.}

12. \textbf{! Бесконечные пределы. Бесконечно большие. Связь между бесконечно малыми и бесконечно большими. Аналоги теорем для бесконечных пределов.}

13. \textbf{Арифметические действия в $\overline{\mathbb{R}}$. Примеры.}

14. \textbf{Неравенство Бернулли.}

15. \textbf{! Определение экспоненты и числа e.}

16. \textbf{Свойства экспоненты.}

17. \textbf{Формула для экспоненты суммы (с леммой).}

18. \textbf{Сравнение скорости возрастания последовательностей $n^k$, $a^n$, $n!$ и $n^n$.}

19. \textbf{Теорема Штольца (для неопределенности $\frac{\infty}{\infty}$). Сумма m-ых степеней натуральных чисел.}

20. \textbf{Теорема Штольца (для неопределенности $\frac{0}{0}$).}

21. \textbf{Подпоследовательности (определение и простейшие свойства). Теорема о стягивающихся отрезках. }

22. \textbf{! Теорема Больцано–Вейерштрасса (в том числе и случай неограниченной последовательности).}

23. \textbf{! Фундаментальные последовательности. Свойства. Критерий Коши.}

24. \textbf{Верхний и нижний пределы. Частичные пределы. Связь между ними.}

25. \textbf{Характеристика верхних и нижних пределов с помощью $N$ и $\epsilon$. Сохранение неравенств для верхних и нижних пределов.}

26. \textbf{! Сходимость рядов. Необходимое условие сходимости рядов. Примеры.}

27. \textbf{Простейшие свойства сходящихся рядов.}

28. \textbf{Окрестности и проколотые окрестности. Предельные точки множества.}

29. \textbf{! Определения предела функций в точке. Простейшие свойства.}

30. \textbf{! Равносильность определения предела по Коши и по Гейне.}

31. \textbf{Свойства функций, имеющих предел.}

32. \textbf{Арифметические действия с пределами.}

33. \textbf{! Теорема о предельном переходе в неравенствах. Теорема о двух милиционерах.}

34. \textbf{! Критерий Коши для предела функций.}

35. \textbf{Левый и правый пределы. Предел монотонной функции.}

\end{document}
