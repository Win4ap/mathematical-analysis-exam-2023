\documentclass[12pt]{article}
\usepackage[utf8]{inputenc}
\usepackage[english,russian]{babel}
\usepackage{graphicx}
\usepackage{amsmath}
\usepackage{amsthm}
\usepackage{caption}
\usepackage{dsfont}
\usepackage{tikz}
\usepackage{amssymb}
\usepackage{subcaption}
\usepackage{imakeidx}
\usepackage{hyperref}
\usepackage[russian]{cleveref}
\usepackage[a4paper,left=15mm,right=15mm,top=30mm,bottom=20mm]{geometry}
\parindent=0mm
\parskip=3mm
\DeclareRobustCommand{\divby}{%
\mathrel{\text{\vbox{\baselineskip.65ex\lineskiplimit0pt\hbox{.}\hbox{.}\hbox{.}}}}%
}

\makeindex
\pagestyle{empty}
\title{Матанализ}
\author{Канта контроль}
\date{\today}
\begin{document}
\maketitle
\large

1. \textbf{Множества: упорядоченная пара, декартово произведение, операции над множествами. Правила де Моргана.}\\
Множество - какой-то набор элементов. Для любого элемента можно сказать принадлежит множеству или нет.\\
$A \subset B$, то есть $\forall x : x \in A \Rightarrow x \in B$ (А - подмножество B)\\
$A = B$, то есть $A \subset B \wedge B \subset A$ (A равно B)\\
$A \subsetneq B$, то есть $A \subset B \wedge A \ne \varnothing \wedge A \ne B$ (A - собственное подмножество B)\\
Способы задать множетсво:
\begin{itemize}
    \item Полное задание: $\{a,b,c\}$.
    \item Неполное: $a_1, a_2, ..., a_k$. Но должно быть понятно как образована последовательно. Например $\{1,5,...,22\}$ — непонятно.
    \item Можно так же и бесконечные: $\{a1, a2, ...\}$.
    \item Словесным описанием. Например, множество простых чисел.
    \item Формулой. Например, пусть задана функция $F(x)$ — функция для всех чисел, которая возращает истину или ложь. Тогда можно взять множество $\{x : F(x)\}$.
\end{itemize}
Операции с множествами:\\
\begin{center}
  \renewcommand{\arraystretch}{1.7}
  \begin{tabular}{| c | c | c |}
    \hline
    \textbf{Символ} & \textbf{Определение} & \textbf{Описание}\\
    \hline
    {\Large $\cap$} & $A \cap B = \{ x \mid x \in A \land x \in B\}$ & Пересечение множеств\\
    \hline
    {\Large $\bigcap_{k=1}^n A_k$} & $A = A_1 \cap A_2 \cap \ldots \cap A_n$ & Пересечение множества множеств \\
    \hline
    {\Large $\cup$} & $A \cup B = \{ x \mid x \in A \lor x \in B\}$ & Объединение множеств\\
    \hline
    {\Large $\bigcup_{k=1}^n A_k$} & $A = A_1 \cup A_2 \cup \ldots \cup A_n$ & Объединение множества множеств \\
    \hline
    {\Large $\setminus$} & $A \setminus B = \{ x \mid x \in A \land x \notin B\}$ & Разность множеств\\
    \hline
    {\Large $\times$} & $A \times B = \{ (x,\,y) \mid x \in A, y \in B\}$ & Декартово произведение\\
    \hline
    {\Large $\bigtriangleup$} & $A \bigtriangleup B = (A \setminus B) \cup (B \setminus A)$ & Симметрическая разность\\
    \hline
    {\Large $\varnothing$} & $\forall x: \; x \notin \varnothing$ & пустое множество\\ 
    \hline
    {\Large $\mathbb{N}$} & & Натуральные числа\\
    \hline
    {\Large $\mathbb{Z}$} & & целые числа \\
    \hline
    {\Large $\mathbb{Q}$} & $\frac{a}{b}$, где $a \in \mathbb{Z}, b \in \mathbb{N}$ & рациональные числа \\
    \hline
    {\Large $\mathbb{R}$} & & действительные числа \\
    \hline
    {\Large $2^X$} & & множество всех подмножеств $X$ \\
    \hline
  \end{tabular}
\end{center}
Важный момент: $1 \in \{1\}$, но  $1 \notin\{\{1\}\}$\\
Правила де Моргана:\\
Пусть есть $A_\alpha \subset X$
\begin{enumerate}
     \item $X \setminus \bigcup_{\alpha \in I} A_\alpha = \bigcap_{\alpha \in I} X \setminus A_\alpha$.
     \item $X \setminus \bigcap_{\alpha \in I} A_\alpha = \bigcup_{\alpha \in I} X \setminus A_\alpha$.
\end{enumerate}
Доказательство: $X \setminus \bigcup_{\alpha \in I} A_{\alpha} = \{x: x \in X \land x \notin A_{\alpha} \; \forall \alpha \in I\} = \{x: \forall \alpha \in I X \setminus A_{\alpha}\} = \bigcap_{\alpha \in I} X \setminus A_{\alpha}$.\\
Упорядоченная пара $\left<x,y\right>$. Важное свойство $\left<x, y\right> = \left<x',y'\right> \iff x = x' \land y = y'$\\
\\
\\

2. \textbf{Отношения: область определения, область значений, обратное отношение, композиция отношений, свойства, примеры.}\\
Отношение $R \subset X \times Y$. $x$ и $y$ находятся в отношении $R$, если их $\left<x, y\right> \in R$.
\begin{itemize}
    \item Область определения $\delta_R = {dom}_R = \{x \in X: \; \exists y \in Y: \; \left<x, y\right> \in R$.
    \item Область значений $\rho_R = {ran}_R = \{y \in Y: \; \exists x \in X: \; \left<x, y\right> \in R$
    \item Обратное отношение $R^{-1} \subset Y \times X \; \; R^{-1} = \{\left<y,x\right>\} \in R$.
    \item Композиция отношения. $R_1 \subset X \times Y, R_2 \subset Y \times Z: \; R_1 \circ R_2 \subset X \times Z$.
    \item $R_1 \circ R_2 = \{\left<x, z\right> \in X \times Z\; \vert \; \exists y \in Y: \; \left<x,y\right> \in R_1 \land \left<y,z\right> \in R_2\}$
\end{itemize}

Свойства:
\begin{itemize}
    \item Функция из $X$ в  $Y$ --- отношение ($\delta_f = X$), для которого верно:
    \[
    \left. \begin{array}{l} \left<x,y\right> \in f \\ \left<x, z\right> \in f \end{array} \right\} \Rightarrow y = z
    .\]
    Используется запись $y = f(y)$. 
    \item Последовательность - функция у которой $\delta_f = \mathbb{N}$
    \item Отношение $R$ называется рефлективным, если $\forall x: \; \left<x, x\right> \in R$.
    \item Отношение $R$ называется симметричным, если  $\forall x, y \in X: \; \left<x, y\right> \in R \Rightarrow \left<y, x\right> \in R$
    \item Отношение $R$ называется иррефлективным, если  $\forall x \left<x,x\right> \notin R$
    \item Отношение $R$ называется антисимметричным, если  $\left. \begin{array}{r} \left<x, y\right> \in R \\ \left<y, x\right> \in R\end{array} \right\} \Rightarrow x = y$
    \item Отношение $R$ называется транзитивным, если  $\left. \begin{array}{r} \left<x, y\right> \in R \\ \left<y, z\right> \in R\end{array} \right\} \Rightarrow \left<x, z\right> \in R$\\
    \item Отношение называется отношением эквивалентности, если отношение рефлективно, симметрично, транзитивно. Например: Равенство, сравнение по модулю $\mathbb{Z}$,  $\|$, отношение подобия треугольников.
    \item Если выполняется рефлективность, антисимметричность и транзитивность, от данное отношение --- отношение нестрогого частичного порядка. Например: $\ge$; $A \subset B$ на $2^X$.
    \item Если выполняется иррефлективность и транзитивность, то данное отношение --- отношение строгого частичного порядка. Например: $>$;  $A$ собственное подмножество  $B$ на  $2^X$.
    \item Иррефлексивность + транзитивность $\Rightarrow$ антисимметрично.
    \item $R$ - нестрогий ч.п.  $\Rightarrow$ $R = \{\left<x,y\right> \in R: \; x \neq y\}$ --- строгий ч.п.
\end{itemize}

Примеры отношений:
\begin{itemize}
    \item Отношение равенства. $R = \{\left<x,x\right>: \; x \in X\}$. Но это просто равенство.
    \item "$\ge$" ($X = \mathbb{R}$). $R = \{ \left<x,y\right>: \; x \ge y\}$
    \item "$>$" ($X = \mathbb{R}$). $R = \{\left<x,y\right>: x > y\}$ \\
        $\delta_{>} = {2,3,4\ldots}$\\
        $\rho_> = \mathbb{N}$\\
        $>^{-1} = < = \{\left<x,y\right>: \; x < t\}$ \\
        $> \circ > = \{\left<x,z\right>\; x-z\ge2\}$
    \item $X$ --- прямые на плоскости. "$\perp$":  $R = \{\left<x,y\right>: \; x \perp y\}$. \\
            $\delta_\perp = \rho_\perp = X$ \\
            $\perp^{-1} = \perp$\\
            $\perp \circ \perp = \|$
    \item $\left<x, y\right> \subset R$, когда  $x$ --- отец  $y$. \\ 
        $\delta_R = \{\text{Все, у кого есть сыновья}\}$. \\
        $\rho_R$ --- религиозный вопрос. См. Библию \\
        $R^{-1} = \text{сын}$ \\
        $R \circ R = \{\text{дед по отцовской линии}\}$
\end{itemize}

3. \textbf{Аксиомы вещественных чисел. Математическая индукция. Существование наибольшего и наименьшего элемента в конечном множестве. Следствия.}\\
Есть две операции:
\begin{itemize}
    \item $+: \mathbb{R} \times \mathbb{R} \to \mathbb{R}$.
        \begin{itemize}
            \item Коммутативность. $x+y=y+x$.
            \item Ассоциативность.  $(x+y)+z=x+(y+z)$
            \item Существует ноль.  $\exists 0 \in \mathbb{R} \; \; x + 0 = x$
            \item Существует противоположный элемент. $\exists (-x) \in \mathbb{R} \; \; x+(-x) = 0$
        \end{itemize}
    \item $\cdot: \mathbb{R} \times \mathbb{R} \to \mathbb{R}$.
        \begin{itemize}
            \item Коммутативность. $x\cdot y=y\cdot x$.
            \item Ассоциативность.  $(x\cdot y)\cdot z=x\cdot (y\cdot z)$
            \item Существует единица.  $\exists 1 \in \mathbb{R} \; \; x \cdot 1 = x$
            \item Существует обратный элемент. $\exists x^{-1} \in \mathbb{R} \; \; x \cdot x^{-1} = 1$
        \end{itemize}
\end{itemize}
Свойство дистрибутивности: $(x+y) \cdot z = x \cdot z + y \cdot z$. Структура с данными операциями называется полем.

Введем отношение $\le$: Оно рефлексивно, антисимметрично и транизитивно, то есть нестрогий частичного порядка. Причем:
\begin{itemize}
    \item $x< y \Rightarrow x+z < y+z$
    \item  $0 \le x \land 0 \le y \Rightarrow 0 \le x\cdot y$
\end{itemize}
Аксиома полноты: Если $A$ и $B \subset \mathbb{R}$ и $\forall a \in A, b \in B: \; a \le b$ и $A \neq \varnothing \land B \neq \varnothing$, тогда $\exists c \in \mathbb{R}\; a \le c \le b$.\\
Множество рациональных не удовлетворяет аксиоме полноты.\\
Например: $A = \{x \in \mathbb{Q} \; \vert \; x^2 < 2\}$,  $B = \{x \in \mathbb{Q} \; \vert \; x>0 \land x^2 > 2\}$. Единственная точка, между этими множествами --- $\sqrt{2}$ 

Пусть $P_n$ - последовательность утверждений. Тогда, если  $P_1$ --- верное и из того, что $P_n$ --- верно следует, что  $P_{n+1}$ --- верно. Тогда все $P_n$ верны  $\forall n \in \mathbb{N}$

Наибольшие/наименьшие элементы:
В непустом конечном множестве $A$ есть наибольший и наименьший элементы.\\
Доказательство. Докажем по индукции:
\begin{itemize}
    \item База. $|A| = 1$. Очевидно.
    \item Переход.  $n \to n + 1$.
    \item Доказательство. Рассмотрим множество из  $n + 1$ элемента  $\{x_1\ldots x_n,x_{n+1}\}$. Выкинем из него последний элемент. Тогда по индукционному предположению у нас есть максимальный элемент $x_k$. Тогда рассмотрим два случая:
         \begin{enumerate}
             \item $x_k \ge x_{n+1}$. Тогда $x_k$ --- наибольший элемент множества $\{x_1\ldots x_n,x_{n+1}\}$.
             \item $x_k < x_{n+1}$. Тогда по транзитивности  $x_{n+1}$ больше всех других элементов множества. Значит, $x_{n+1}$ --- наибольший элемент множества $\{x_1\ldots x_n,x_{n+1}\}$.
        \end{enumerate}
\end{itemize}

Определение. Пусть $A \subset \mathbb{R}$. Тогда  $A$ --- ограничено сверху, если  $\exists c \in \mathbb{R}: \; \forall a \in A\; a \le c$. Такое $c$ называется верхней границей.\\

Определение. Пусть $A \subset \mathbb{R}$. Тогда  $A$ --- ограничено снизу, если  $\exists b \in \mathbb{R}: \; \forall a \in A\; a \ge b$. Такое  $b$ называется нижней границей.\\

Определение. Пусть $A \subset \mathbb{R}$. Тогда  $A$ --- ограничено, если оно ограничено сверху и снизу. Например: $\mathbb{N}$ не ограничено сверху, но ограничено снизу.\\

Доказательсво. Пусть $\exists c \in \mathbb{R}: \; c \ge n\; \forall n \in \mathbb{N}$. Тогда это противоречит принципу Архимеда при $x = c, y = 1$. \\ Для ограниченности снизу достаточно взять $c=-1$.\\

4. \textbf{Принцип Архимеда. Следствия. Наибольший элемент в множестве целых чисел. Существование целой части числа.}\\
Принцип Архимеда:\\
Пусть $x \in \mathbb{R} \land y > 0$. Тогда  $\exists n \in \mathbb{N}:\; x < ny$\\
Доказательство:\\
$A = \{u \in \mathbb{R}:\; \exists n \in \mathbb{N}:\; u < ny\}$. Пусть $A \neq! \mathbb{R}$, $B = \mathbb{R} \setminus A \neq \varnothing$,  $A \neq \varnothing$, т.к.  $0 \in A$.\\
Возьмем  $a \in A, b \in B$.  $b < a \Rightarrow \exists n: \; a < ny \Rightarrow b < ny \Rightarrow \texttt{противоречие}$. \\
По аксиоме полноты $\exists c \in \mathbb{R}: \; a \le c \le b \; \forall a \in A, \forall b \in B$.\\
Пусть $c \in A$. Тогда  $c < ny \Rightarrow c < c + y < ny + y = (n+1)y \Rightarrow c < c + y \Rightarrow c + y \in A$. Противоречие.\\
Пусть  $c \in B$.  Рассмотрим $c-y<c \Rightarrow c-y \in A \Rightarrow \exists n: \; c - y < ny \Rightarrow c < ny + y = (n+1)y \Rightarrow c \in A$. Противоречие.\\

Следствие:\\
Если $\epsilon > 0$, то $\exists n \in \mathbb{N} \; \frac{1}{n} < \epsilon$\\
Доказательство:\\
$x = 1, y = \epsilon \Rightarrow ny = n \epsilon > x = 1 \iff \epsilon > \frac{1}{n}$\\

Теорема:\\
В непустом ограниченном сверху (снизу) множестве целых чисел есть наибольший (наименьший) элемент.\\
Доказательство:\\
Пусть $A \subset \mathbb{Z}$.  $c$ --- его верхняя граница.\\
Возьмем  $b \in A$ и рассмотрим  $B := {x \in A \mid x \ge b}$. Заметим, что $B$ содержит конечное число элементов, значит в нем есть наибольший элемент. Пусть это $m \in B$:  $\forall x \in B:\; x \le m$. Докажем, что $m$ --- наибольший элемент и в  $A$.\\
Для этого заметим, что любой $x \in A$ либо лежит в $B$, либо  $x < b$, а по транзитивности  $x < b \le m$.\\

Пусть $x \in \mathbb{R}$, тогда  $[x] = \lfloor x \rfloor$ --- наименьшее целое число, не превосходящее  $x$.
\begin{enumerate}
    \item $[x]  \le x < [x] + 1$\\
        Левое неравенство очевидно. Правое неравенство можно доказать от противного: пусть $x \ge [x] + 1$, тогда справа целое число большое $[x]$, но меньшее  $x$. Противоречие.
    \item $x - 1 < [x] \le x$
\end{enumerate}

5. \textbf{! Супремум и инфимум. Определение и теорема существования. Характеристика супремума.}

6. \textbf{! Теорема о вложенных отрезках. Существенность условий.}

7. \textbf{! Монотонные и ограниченные последовательности. Два определения предела и их равносильность. Примеры.}

8. \textbf{! Простейшие свойства пределов последовательностей (единственность предела, предельный переход в неравенстве, ограниченность).}

9. \textbf{! Теорема о стабилизации знака и теорема о двух милиционерах. Следствия.}

10. \textbf{! Предел монотонной последовательности.}

11. \textbf{Арифметические свойства пределов последовательности.}

12. \textbf{! Бесконечные пределы. Бесконечно большие. Связь между бесконечно малыми и бесконечно большими. Аналоги теорем для бесконечных пределов.}

13. \textbf{Арифметические действия в $\overline{\mathbb{R}}$. Примеры.}

14. \textbf{Неравенство Бернулли.}

15. \textbf{! Определение экспоненты и числа e.}

16. \textbf{Свойства экспоненты.}

17. \textbf{Формула для экспоненты суммы (с леммой).}

18. \textbf{Сравнение скорости возрастания последовательностей $n^k$, $a^n$, $n!$ и $n^n$.}

19. \textbf{Теорема Штольца (для неопределенности $\frac{\infty}{\infty}$). Сумма m-ых степеней натуральных чисел.}

20. \textbf{Теорема Штольца (для неопределенности $\frac{0}{0}$).}

21. \textbf{Подпоследовательности (определение и простейшие свойства). Теорема о стягивающихся отрезках. }

22. \textbf{! Теорема Больцано–Вейерштрасса (в том числе и случай неограниченной последовательности).}

23. \textbf{! Фундаментальные последовательности. Свойства. Критерий Коши.}

24. \textbf{Верхний и нижний пределы. Частичные пределы. Связь между ними.}

25. \textbf{Характеристика верхних и нижних пределов с помощью $N$ и $\epsilon$. Сохранение неравенств для верхних и нижних пределов.}

26. \textbf{! Сходимость рядов. Необходимое условие сходимости рядов. Примеры.}

27. \textbf{Простейшие свойства сходящихся рядов.}

28. \textbf{Окрестности и проколотые окрестности. Предельные точки множества.}

29. \textbf{! Определения предела функций в точке. Простейшие свойства.}

30. \textbf{! Равносильность определения предела по Коши и по Гейне.}

31. \textbf{Свойства функций, имеющих предел.}

32. \textbf{Арифметические действия с пределами.}

33. \textbf{! Теорема о предельном переходе в неравенствах. Теорема о двух милиционерах.}

34. \textbf{! Критерий Коши для предела функций.}

35. \textbf{Левый и правый пределы. Предел монотонной функции.}

\end{document}
